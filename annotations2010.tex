% \cite[p.41]{ecra07}

\documentclass{article}

\begin{document}
\title{TODO}
\author[UMN]{Garrett Holmstrom}
\ead{garrett@cs.umn.edu}

\address[UMN]{Dept. of Computer Science and Engineering,
    University of Minnesota, 4-192 EE/CS Bldg., 200 Union St SE, Minneapolis,
    MN 55455, USA.}

\maketitle

\begin{abstract}

\end{abstract}

\section{Introduction}

\subsection{MinneTAC architecture}

% TODO:  basic background information about MinneTAC goes here

MinneTAC is designed to minimize coupling between its components to support independent work on multiple lines of research.  To that end we use a component-oriented approach...

A MinneTAC agent consists of a set of components for each major decision process in the TAC-SCM game:  \textsc{Procurement}, \textsc{Production}, \textsc{Sales}, and \textsc{Shipping}.  All data to be shared among components are kept in the Repository, which plays the role of the Blackboard in the \emph{Blackboard} pattern\cite{Busch96}.  The \textsc{Communications} component manages interaction with the game server.  Finally, the \textsc{Oracle} component contains a large number of smaller components that maintain models of markets and inventory and perform analyses and predictions.  Ideally, each of these components depends solely upon the \textsc{Repository}, which completely separates major decision processes and allowing researchers to work on them independently.

% TODO:  hub-and-spoke diagram

The \textsc{Oracle} component contains several hundred classes that make up the majority of MinneTAC's code base.

\subsection{Previous work}

\end{document}
