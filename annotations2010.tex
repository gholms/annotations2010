\documentclass{article}

\begin{document}
\title{TODO}
\author[UMN]{Garrett Holmstrom}
\ead{garrett@cs.umn.edu}

\address[UMN]{Dept. of Computer Science and Engineering,
    University of Minnesota, 4-192 EE/CS Bldg., 200 Union St SE, Minneapolis,
    MN 55455, USA.}

\maketitle

\begin{abstract}

\end{abstract}

\section{Introduction}

% TODO:  basic information about TAC-SCM?

\subsection{MinneTAC architecture}

MinneTAC is designed to minimize coupling between its components to support independent work on multiple lines of research.  To that end we use a component-oriented approach...
% TODO:  finish above

A MinneTAC agent consists of a set of components for each major decision process in the TAC-SCM game:  \textsc{Procurement}, \textsc{Production}, \textsc{Sales}, and \textsc{Shipping}.  All data to be shared among components are kept in the Repository, which plays the role of the Blackboard in the \emph{Blackboard} pattern\cite{Busch96}.  The \textsc{Communications} component manages interaction with the game server.  Finally, the \textsc{Oracle} component contains a large number of smaller components that maintain models of markets and inventory and perform analyses and predictions.  Ideally, each of these components depends solely upon the \textsc{Repository}, which completely separates major decision processes and allowing researchers to work on them independently.

% TODO:  hub-and-spoke diagram

Since decision components cannot depend on one another, they communicate using \emph{evaluations} that are accessible through the various data elements in the \textsc{Repository}.  Any calculations or analyses that are performed on \textsc{Repository} data can be encapsulated in the form of evaluations and made available to all other components via the \textsc{Repository}.  The \textsc{Oracle} component contains a large number of configurable \emph{evaluator} classes that perform analyses on \textsc{Repository} data.

All the major data elements in the \textsc{Repository}, such as RFQs, offers, components, and so forth, are Evaluable types.  Each Evaluable can be queried for related Evaluations by passing it the name of the needed evaluation.  An EvaluationFactory maintains a mapping of Evaluation names to Evaluator instances, and calls upon Evaluators to produce Evaluations on demand.  Evaluators can back-chain by requesting other Evaluations as they attempt to produce their results.  By this means, an Evaluation may be composed from several other Evaluations that are in turn generated by their own Evaluators.

The resulting \textsc{Oracle} component is essentially a framework for a set of small, configurable sub-components from which other components can request analyses and predictions.  Most of these sub-components are Evaluators, though other types also exist.  The \textsc{Oracle} itself merely uses its configuration data to create and configure instances of Evaluators and other subclasses of ConfiguredObject.  \emph{ConfiguredObject} is an abstract class whose instances have names and some ability to configure themselves, given an appropriate clause from a XML configuration file.  The \textsc{Oracle} creates ConfiguredObject instances and keeps track of them by mapping their names as given in the configuration file to instances.

% TODO:  UML diagram?

During initialization, the \textsc{Oracle} processes a configuration clause that specifies which instances to create, and for each such instance, what name to give it, the values of any parameters necessary to configure it, and the names of any instances it queries for input data.  Figure~\ref{fig:xconf-simpleprice} illustrates such a configuration clause for an evaluator with one parameter and four inputs.

\begin{figure}[ht]
\begin{center}
\fbox{\begin{minipage}
\begin{small}
\begin{tabbing}
\texttt{<evaluator class="edu.umn.cs.tac.oracle.eval.SimplePriceEvaluator"} \\
\>\>\>\>\texttt{name="simple-price">} \\
\>\>\>\>\texttt{<parameters price-probability-exponent="1.0"/>} \\
\texttt{</evaluator>} \\
\texttt{<graph source="simple-price">} \\
\>\>\>\>\texttt{<quantity source="customer-quantity"/>} \\
\>\>\>\>\texttt{<effective-demand source="effective-demand"/>} \\
\>\>\>\>\texttt{<allocation source="allocation"/>} \\
\>\>\>\>\texttt{<regression source="probability-of-acceptance"/>} \\
\texttt{</graph>}
\end{tabbing}
\end{small}
\end{minipage}}
\end{center}
\caption{Configuration clause for a price evaluator that uses one parameter and several inputs}
\label{fig:xconf-simpleprice}
\end

Taken together, a set of Evaluators and other ConfiguredObjects comprises a directed graph of modular components that constitute the majority of the functional code in a MinneTAC configuration.  MinneTAC's modular architecture allows one to easily substitute components for one another for comparative purposes or to create an entirely new set of components with only minimal knowledge of how the agent works internally.  Additionally, using configuration files to defnie graphs of components allows developers to add new components without disrupting other developers' configurations.

\subsection{Design challenges}

\subsection{Previous work}

\end{document}
